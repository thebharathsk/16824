\documentclass[11pt,addpoints,answers]{exam}
\usepackage[margin=1in]{geometry}
\usepackage{amsmath, amsfonts}
\usepackage{enumerate}
\usepackage{graphicx}
\usepackage{titling}
\usepackage{url}
\usepackage{xfrac}
\usepackage{geometry}
\usepackage{graphicx}
\usepackage{natbib}
\usepackage{amsmath}
\usepackage{amssymb}
\usepackage{amsthm}
\usepackage{paralist}
\usepackage{epstopdf}
\usepackage{tabularx}
\usepackage{longtable}
\usepackage{multirow}
\usepackage{multicol}
\usepackage[colorlinks=true,urlcolor=blue]{hyperref}
\usepackage{fancyvrb}
\usepackage{algorithm}
\usepackage{algorithmic}
\usepackage{float}
\usepackage{paralist}
\usepackage[svgname]{xcolor}
\usepackage{enumerate}
\usepackage{array}
\usepackage{times}
\usepackage{url}
\usepackage{comment}
\usepackage{environ}
\usepackage{times}
\usepackage{textcomp}
\usepackage{caption}
\usepackage[colorlinks=true,urlcolor=blue]{hyperref}
\usepackage{listings}
\usepackage{parskip} % For NIPS style paragraphs.
\usepackage[compact]{titlesec} % Less whitespace around titles
\usepackage[inline]{enumitem} % For inline enumerate* and itemize*
\usepackage{datetime}
\usepackage{comment}
% \usepackage{minted}
\usepackage{lastpage}
\usepackage{color}
\usepackage{xcolor}
\usepackage{listings}
\usepackage{tikz}
\usetikzlibrary{shapes,decorations,bayesnet}
%\usepackage{framed}
\usepackage{graphicx}
\usepackage{booktabs}
\usepackage{cprotect}
\usepackage{xcolor}
\usepackage{verbatimbox}
\usepackage[many]{tcolorbox}
\usepackage{cancel}
\usepackage{wasysym}
\usepackage{mdframed}
\usepackage{subcaption}
\usetikzlibrary{shapes.geometric}

%%%%%%%%%%%%%%%%%%%%%%%%%%%%%%%%%%%%%%%%%%%
% Formatting for \CorrectChoice of "exam" %
%%%%%%%%%%%%%%%%%%%%%%%%%%%%%%%%%%%%%%%%%%%

\CorrectChoiceEmphasis{}
\checkedchar{\blackcircle}

%%%%%%%%%%%%%%%%%%%%%%%%%%%%%%%%%%%%%%%%%%%
% Better numbering                        %
%%%%%%%%%%%%%%%%%%%%%%%%%%%%%%%%%%%%%%%%%%%

\numberwithin{equation}{section} % Number equations within sections (i.e. 1.1, 1.2, 2.1, 2.2 instead of 1, 2, 3, 4)
\numberwithin{figure}{section} % Number figures within sections (i.e. 1.1, 1.2, 2.1, 2.2 instead of 1, 2, 3, 4)
\numberwithin{table}{section} % Number tables within sections (i.e. 1.1, 1.2, 2.1, 2.2 instead of 1, 2, 3, 4)


%%%%%%%%%%%%%%%%%%%%%%%%%%%%%%%%%%%%%%%%%%%
% Common Math Commands                    %
%%%%%%%%%%%%%%%%%%%%%%%%%%%%%%%%%%%%%%%%%%%
\input{mathabbreviations.tex}

%%%%%%%%%%%%%%%%%%%%%%%%%%%%%%%%%%%%%%%%%%%
% Code highlighting with listings         %
%%%%%%%%%%%%%%%%%%%%%%%%%%%%%%%%%%%%%%%%%%%

\definecolor{bluekeywords}{rgb}{0.13,0.13,1}
\definecolor{greencomments}{rgb}{0,0.5,0}
\definecolor{redstrings}{rgb}{0.9,0,0}
\definecolor{light-gray}{gray}{0.95}

\newcommand{\MYhref}[3][blue]{\href{#2}{\color{#1}{#3}}}%

\definecolor{dkgreen}{rgb}{0,0.6,0}
\definecolor{gray}{rgb}{0.5,0.5,0.5}
\definecolor{mauve}{rgb}{0.58,0,0.82}

\lstdefinelanguage{Shell}{
  keywords={tar, cd, make},
  %keywordstyle=\color{bluekeywords}\bfseries,
  alsoletter={+},
  ndkeywords={python, py, javac, java, gcc, c, g++, cpp, .txt, octave, m, .tar},
  %ndkeywordstyle=\color{bluekeywords}\bfseries,
  identifierstyle=\color{black},
  sensitive=false,
  comment=[l]{//},
  morecomment=[s]{/*}{*/},
  commentstyle=\color{purple}\ttfamily,
  stringstyle=\color{red}\ttfamily,
  morestring=[b]',
  morestring=[b]",
  backgroundcolor = \color{light-gray}
}

\lstset{columns=fixed, basicstyle=\ttfamily,
    backgroundcolor=\color{light-gray},xleftmargin=0.5cm,frame=tlbr,framesep=4pt,framerule=0pt}



%%%%%%%%%%%%%%%%%%%%%%%%%%%%%%%%%%%%%%%%%%%
% Custom box for highlights               %
%%%%%%%%%%%%%%%%%%%%%%%%%%%%%%%%%%%%%%%%%%%

% Define box and box title style
\tikzstyle{mybox} = [fill=blue!10, very thick,
    rectangle, rounded corners, inner sep=1em, inner ysep=1em]

% \newcommand{\notebox}[1]{
% \begin{tikzpicture}
% \node [mybox] (box){%
%     \begin{minipage}{\textwidth}
%     #1
%     \end{minipage}
% };
% \end{tikzpicture}%
% }

\NewEnviron{notebox}{
\begin{tikzpicture}
\node [mybox] (box){
    \begin{minipage}{\textwidth}
        \BODY
    \end{minipage}
};
\end{tikzpicture}
}

%%%%%%%%%%%%%%%%%%%%%%%%%%%%%%%%%%%%%%%%%%%
% Commands showing / hiding solutions     %
%%%%%%%%%%%%%%%%%%%%%%%%%%%%%%%%%%%%%%%%%%%

%% To HIDE SOLUTIONS (to post at the website for students), set this value to 0: \def\issoln{0}
\def\issoln{0}
% Some commands to allow solutions to be embedded in the assignment file.
\ifcsname issoln\endcsname \else \def\issoln{0} \fi
% Default to an empty solutions environ.
\NewEnviron{soln}{}{}
% Default to an empty qauthor environ.
\NewEnviron{qauthor}{}{}
% Default to visible (but empty) solution box.
\newtcolorbox[]{studentsolution}[1][]{%
    breakable,
    enhanced,
    colback=white,
    title=Solution,
    #1
}

\if\issoln 1
% Otherwise, include solutions as below.
\RenewEnviron{soln}{
    \leavevmode\color{red}\ignorespaces
    \textbf{Solution} \BODY
}{}
\fi

\if\issoln 1
% Otherwise, include solutions as below.
\RenewEnviron{solution}{}
\fi

%%%%%%%%%%%%%%%%%%%%%%%%%%%%%%%%%%%%%%%%%%%
% Commands for customizing the assignment %
%%%%%%%%%%%%%%%%%%%%%%%%%%%%%%%%%%%%%%%%%%%

\newcommand{\courseNum}{\href{https://visual-learning.cs.cmu.edu/}{16824}}
\newcommand{\courseName}{\href{https://visual-learning.cs.cmu.edu/}{Visual Learning and Recognition}}
\newcommand{\courseSem}{\href{https://visual-learning.cs.cmu.edu/}{Spring 2023}}
\newcommand{\courseUrl}{\url{https://piazza.com/class/lcy4ow5l5xp2fl}}
\newcommand{\hwNum}{Homework 3}
\newcommand{\hwTopic}{Transformers}
\newcommand{\hwName}{\hwNum: \hwTopic}
\newcommand{\outDate}{Sun, 19th March 2023}
\newcommand{\dueDate}{Mon, 3rd April 2023}
\newcommand{\instructorName}{Deepak Pathak}
\newcommand{\taNames}{Ananye Agarwal, Rohan Choudhury, Murtaza Dalal, Russell Mendonca}

%\pagestyle{fancyplain}
\lhead{\hwName}
\rhead{\courseNum}
\cfoot{\thepage{} of \numpages{}}

\title{\textsc{\hwName}} % Title


\author{}

\date{}

%%%%%%%%%%%%%%%%%%%%%%%%%%%%%%%%%%%%%%%%%%%%%%%%%
% Useful commands for typesetting the questions %
%%%%%%%%%%%%%%%%%%%%%%%%%%%%%%%%%%%%%%%%%%%%%%%%%

\newcommand \expect {\mathbb{E}}
\newcommand \mle [1]{{\hat #1}^{\rm MLE}}
\newcommand \map [1]{{\hat #1}^{\rm MAP}}
\newcommand \argmax {\operatorname*{argmax}}
\newcommand \argmin {\operatorname*{argmin}}
\newcommand \code [1]{{\tt #1}}
\newcommand \datacount [1]{\#\{#1\}}
\newcommand \ind [1]{\mathbb{I}\{#1\}}

\newcommand{\blackcircle}{\tikz\draw[black,fill=black] (0,0) circle (1ex);}
\renewcommand{\circle}{\tikz\draw[black] (0,0) circle (1ex);}

\newcommand{\pts}[1]{\textbf{[#1 pts]}}

%%%%%%%%%%%%%%%%%%%%%%%%%%
% Document configuration %
%%%%%%%%%%%%%%%%%%%%%%%%%%

% Don't display a date in the title and remove the white space
\predate{}
\postdate{}
\date{}

%%%%%%%%%%%%%%%%%%
% Begin Document %
%%%%%%%%%%%%%%%%%%


\begin{document}

\section*{}
\begin{center}
  \textsc{\LARGE \hwNum} \\
%   \textsc{\LARGE \hwTopic\footnote{Compiled on \today{} at \currenttime{}}} \\
  \vspace{1em}
  \textsc{\large \courseNum{} \courseName{} (\courseSem)} \\
  %\vspace{0.25em}
  \courseUrl\\
  \vspace{1em}
  RELEASED: \outDate \\
  DUE: \dueDate \\
  Instructor: \instructorName \\
  TAs: \taNames
\end{center}

\section*{START HERE: Instructions}
\begin{itemize}
\item \textbf{Collaboration policy:} All are encouraged to work together BUT you must do your own work (code and write up). If you work with someone, please include their name in your write-up and cite any code that has been discussed. If we find highly identical write-ups or code or lack of proper accreditation of collaborators, we will take action according to strict university policies. See the \href{hhttps://www.cmu.edu/policies/student-and-student-life/academic-integrity.html}{Academic Integrity Section} detailed in the initial lecture for more information.

\item\textbf{Late Submission Policy:} There are a \textbf{total of 7} late days across all homework submissions. Submissions more than 7 days after the deadline will receive a 0.

\item\textbf{Submitting your work:}

\begin{itemize}

\item We will be using Gradescope (\url{https://gradescope.com/}) to submit the Problem Sets. Please use the provided template only. Submissions must be written in LaTeX. All submissions not adhering to the template will not be graded and receive a zero. 
\item \textbf{Deliverables:} Please submit all the \texttt{.py} files. Add all relevant plots and text answers in the boxes provided in this file. TO include plots you can simply modify the already provided latex code. Submit the compiled \texttt{.pdf} report as well.
\end{itemize}
\end{itemize}
\emph{NOTE: Partial points will be given for implementing parts of the homework even if you don't get the mentioned numbers as long as you include partial results in this pdf.}
\clearpage

\section{Image Captioning with Transformers (70 points)}



We will be implementing the different pieces of a Transformer decoder (\href{https://arxiv.org/pdf/1706.03762.pdf}{Transformers}), and train it for image captioning on a subset of the \href{https://cocodataset.org/#home}{COCO dataset}. 
\begin{itemize}
    \item \textbf{Setup:} Run the following command to extract COCO data, in the \texttt{transformer\_captioning/datasets} folder : \texttt{./get\_coco\_captioning.sh}
    %: \\ 
    % \texttt{get_coco_captioningsh}
\item \textbf{Question:} Follow the instructions in the \texttt{README.md} file in the \texttt{transformer\_captioning} folder to complete the implementation of the transformer decoder.
\item \textbf{Deliverables:} After implementing all parts, use run.py for training the full model. The code will log plots to \texttt{plots}. Extract plots and paste them into the appropriate section below. 

\item \textbf{Expected results:}
    These are expected training losses after 100 epochs. Do not change the seed in run.py.
    \begin{itemize}
        \item 2-heads, 2-layers, lr 1e-4: Final loss $\leq$ 1 
        \item 4-heads, 6-layers, lr 1e-4: Final loss $\leq$ 0.3 
        \item 4-heads, 6-layers, lr 1e-3: Final loss $\leq$ 0.05
    \end{itemize}
\end{itemize}

\begin{questions}
\question Paste training loss plots for each of the three hyper-param configs
\\
2-heads-2-layers-lr-1e-4: \textbf{0.522527} \\
4-heads-6-layers-lr-1e-4: \textbf{0.148806} \\
4-heads-6-layers-lr-1e-3: \textbf{0.054783} 
\begin{figure}[H]
    \centering
    % TODO: put your plot here.
    \begin{subfigure}[b]{0.32\linewidth}
        \includegraphics[width=\linewidth]{./results/transformer_captioning/results_1/case1_loss_out.png}
        \caption{2-heads-2-layers-lre-4}
    \end{subfigure}
    \begin{subfigure}[b]{0.32\linewidth}
        \includegraphics[width=\linewidth]{./results/transformer_captioning/results_2/case1_loss_out.png}
        \caption{4-heads-6-layers-lre-4}
    \end{subfigure}
    \begin{subfigure}[b]{0.32\linewidth}
        \includegraphics[width=\linewidth]{./results/transformer_captioning/results_3/case1_loss_out.png}
        \caption{4-heads-6-layers-lre-3}
    \end{subfigure}
    
\end{figure}
\question Paste any three generated captioning samples from the training set. The provided code creates these plots at the end of training.
\\
\begin{figure}[H]
    \centering
    \begin{subfigure}[b]{0.32\linewidth}
        \includegraphics[width=\linewidth]{./results/transformer_captioning/results_3/case1_train_0.png}
        \caption{Sample1}
    \end{subfigure}
    \begin{subfigure}[b]{0.32\linewidth}
        \includegraphics[width=\linewidth]{./results/transformer_captioning/results_3/case1_train_3.png}
        \caption{Sample2}
    \end{subfigure}
    \begin{subfigure}[b]{0.32\linewidth}
        \includegraphics[width=\linewidth]{./results/transformer_captioning/results_3/case1_train_4.png}
        \caption{Sample3}
    \end{subfigure}
\end{figure}
%\question 
% What would you change in the training procedure to get better validation performance? : \\ \textbf{TODO: answer.} 

\end{questions}
\clearpage

\section{Classification with Vision Transformers (30 points)}



We will use the transformer you implemented in the previous part to implement a Vision Transformer (\href{https://arxiv.org/abs/2010.11929}{ViT}), for classification on CIFAR10. 

\begin{itemize}
    \item \textbf{Question:} Follow the instructions in the \texttt{README.md} file in the \texttt{vit\_classification} folder. You are encouraged to resuse code from the previous question. 
    \item \textbf{Deliverables:} Run training using \texttt{run.py} for training the full model. The code will log plots \texttt{acc\_out.png} (train and test accuracy) and \texttt{loss\_out.png} (train loss). 
    \item \textbf{Expected Results:} After 100 epochs, test accuracy should be $\approx 68\%$, train accuracy should be $\approx 100\%$, and training loss $\approx 0.25$. 
\end{itemize}

\begin{figure}[H]
    \centering
    \begin{subfigure}[b]{0.32\linewidth}
        \includegraphics[width=\linewidth]{./results/vit_classification/acc_out.png}
        \caption{Train/test accuracy}
    \end{subfigure}
    \begin{subfigure}[b]{0.32\linewidth}
        \includegraphics[width=\linewidth]{./results/vit_classification/loss_out.png}
        \caption{Training loss}
    \end{subfigure}
\end{figure}

\clearpage

\textbf{Collaboration Survey} Please answer the following:

\begin{enumerate}
    \item Did you receive any help whatsoever from anyone in solving this assignment?
    \begin{checkboxes}
     \choice Yes
     \CorrectChoice No
    \end{checkboxes}
    \begin{itemize}
        \item If you answered `Yes', give full details:
        \item (e.g. “Jane Doe explained to me what is asked in Question 3.4”)
    \end{itemize}

    \begin{tcolorbox}[fit,height=3cm,blank, borderline={1pt}{-2pt},nobeforeafter]
    %Input your solution here.  Do not change any of the specifications of this solution box.
    \end{tcolorbox}

    \item Did you give any help whatsoever to anyone in solving this assignment?
    \begin{checkboxes}
     \choice Yes
     \CorrectChoice No\
    \end{checkboxes}
    \begin{itemize}
        \item If you answered `Yes', give full details:
        \item (e.g. “I pointed Joe Smith to section 2.3 since he didn’t know how to proceed with Question 2”)
    \end{itemize}

    \begin{tcolorbox}[fit,height=3cm,blank, borderline={1pt}{-2pt},nobeforeafter]
    %Input your solution here.  Do not change any of the specifications of this solution box.
    \end{tcolorbox}

    \item Note that copying code or writeup even from a collaborator or anywhere on the internet violates the \href{hhttps://www.cmu.edu/policies/student-and-student-life/academic-integrity.html}{Academic Integrity Code of Conduct}.
\end{enumerate}

\end{document}